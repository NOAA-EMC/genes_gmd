\documentclass[12pt]{article}

\oddsidemargin=0.4in
\evensidemargin=0.4in
\textwidth=5.7in
\topmargin=-0.6in
\textheight=8.5in

\usepackage{psfig}
\usepackage{natbib}
\usepackage{svn}

\SVN $Revision$

\newcommand{\pstyle}{myheadings}
%\newcommand{\pstyle}{plain}
\newfont{\ff}{cmsl12}

\newcommand{\wwt}{WAVEWATCH III$\:$\textsuperscript\textregistered}
\newcommand{\ww}{WAVEWATCH III}
\newcommand{\ws}{WW3}
\newcommand{\gmd}{GMD}
\newcommand{\refs}{(\ldots references \ldots)}
\newcommand{\p}{\partial}
\newcommand{\degree}{^{\circ}}
\newcommand{\bk}{\mbox{\boldmath $k$}}
\newcommand{\bc}{\mbox{\boldmath $c$}}
\newcommand{\lae}{$\lambda =$}
\newcommand{\mue}{$\mu =$}
\newcommand{\CCe}{$C =$}

\newcommand{\file}{\sf}
\newcommand{\code}{\tt}
\newcommand{\dir}{\sf}
\newcommand{\F}{\sc}

\newcommand{\pb}{\strut \vfill \pagebreak}
\newcommand{\bpage}{\vfill \pagebreak \strut

\vspace{2.5in} \centerline{This page is intentionally left blank.}}
\newcommand{\bpagea}{\strut

\vspace{2.5in} \centerline{This page is intentionally left blank.}}

\newcommand{\newsec}{\setcounter{equation}{0}
                      \setcounter{myfigno}{0}
                      \setcounter{mytabno}{0}}

\newenvironment{flist}{\begin{list}{nofile ?}{\parsep 0mm
            \itemsep 0mm \leftmargin 35mm \labelwidth 25mm
            \rightmargin 10mm}}{\end{list}}
\newenvironment{flist2}{\begin{list}{nofile ?}{\parsep 0mm
            \itemsep 0mm \leftmargin 45mm \labelwidth 35mm
            \rightmargin 10mm}}{\end{list}}
\newcommand{\fit}[2]{\item[{\file{#1}}\hfill]{#2}}

\newenvironment{plist}{\begin{list}{nofile ?}{\parsep 0mm
            \itemsep 0mm \leftmargin 35mm \labelwidth 25mm
            \rightmargin 10mm}}{\end{list}}
\newcommand{\pit}[2]{\item[{\code{#1}}\hfill]{#2}}

\newcounter{myfigno}[section]
\newenvironment{myfig}[1]{\begin{figure}[#1]
                         \refstepcounter{myfigno}}                       
                        {\end{figure}}
\newenvironment{myfigx}[1]{\begin{figure}[#1]}                       
                        {\end{figure}}
\newcommand{\myfcap}[1]{\begin{list}{\ff Fig. \themyfigno\ :~\hfill}
                       {\rightmargin 8mm \labelsep 0mm
                        \labelwidth 8mm \leftmargin 8mm
                        \topsep 0mm \parskip 0mm \partopsep 0mm }
                        \item \ff #1 \end{list}}
\newcommand{\myfcapc}[1]{\begin{center} \ff Fig. \themyfigno\ :~ #1
                         \end{center}}
\newcommand{\myfcapx}[1]{\begin{center} \ff #1 \end{center}}

\newcounter{mytabno}[section]
\newenvironment{mytab}[1]{\begin{table}[#1]
                         \refstepcounter{mytabno}}                       
                        {\end{table}}
\newcommand{\mytcap}[1]{\begin{list}{\ff Table \themytabno :~\hfill}
                       {\rightmargin 8mm \labelsep 0mm
                        \labelwidth 8mm \leftmargin 8mm
                        \topsep 0mm \parskip 0mm \partopsep 0mm }
                        \item \ff #1 \end{list}
                        \vspace{\baselineskip}}
\newcommand{\mytcapc}[1]{\begin{center} \ff Table \themytabno : #1
                         \end{center} \vspace{\baselineskip}}

\renewcommand{\themyfigno}{\thesection.\arabic{myfigno}}
\renewcommand{\themytabno}{\thesection.\arabic{mytabno}}
\renewcommand{\theequation}{\thesection.\arabic{equation}}

\newcounter{mylistno}

\newcommand{\topline}{\rule{137mm}{0.5mm}}
\newcommand{\botline}{\vspace{1 mm}\rule{137mm}{0.5mm}}

\begin{document}

%--------------------------------------------------------------%
%           Title page                                         %
%--------------------------------------------------------------%

\pagestyle{empty}

\strut \vspace{5mm}

\begin{center} 
U. S. Department of Commerce \\
National Oceanic and Atmospheric Administration \\
National Weather Service \\
National Centers for Environmental Prediction \\
5200 Auth Road Room 207 \\
Camp Springs, MD 20746

\vspace{15mm}

{\bf Technical Note}

\vspace{15mm}

{\large A genetic optimization package for the Generalized \\ Multiple DIA in
  \wwt\ $^\dag$.}

\vspace{20mm}

Hendrik l. Tolman$^\ddag$
\\
Environmental Modeling Center \\
Marine Modeling and Analysis Branch

\vspace{25mm}

\today\ (DRAFT, Revision: \SVNRevision)
%Version 1.0, December 2010

\vfill {\sc this is an unreviewed manuscript, primarily
intended for informal exchange of information among ncep staff
members}

\end{center}
\noindent \rule{140mm}{0.5mm} \\
{\small $^\dag$ MMAB Contribution No.~289. \\
$^\ddag$ e-mail: Hendrik.Tolman@NOAA.gov} \\

\bpage

\pagebreak

\markright{\fbox{\rm{DRAFT}} \hspace{2mm} \today}
\pagestyle{\pstyle}
\pagenumbering{roman}
\setcounter{page}{1}

%--------------------------------------------------------------%
%            Abstract                                          %
%--------------------------------------------------------------%
\addtocontents{toc}{\vspace{\baselineskip}}
\addcontentsline{toc}{subsection}{Abstract}

\begin{abstract}
This report describes a genetic optimization package for the Generalized
Multiple DIA (GMD) in the \ww\ modeling framework. This report will be updated
as needed, depending upon development of this package or of the underlying
wave model.
\end{abstract}

\vspace{\baselineskip}
\vspace{\baselineskip}
\vspace{\baselineskip}

\begin{center}
{\bf Change log} \\
\vspace{\baselineskip}
\begin{tabular}{|c|c|c|c|l|} \hline
ver. & WW   & rev. $^*$   & date    & comment    \\ \hline \hline
 1.0 & 3.15 &  11596 & December 23, 2010 & Initial MMAB No. 289.        \\
     &      &        &                   & Experimental WW version      \\
     &      &        &                   & used for \cite{tol:MMAB10d}. \\
 1.1 & 3.15 & \SVNRevision & \today      & Draft work version           \\
\hline \end{tabular}
\end{center}
\hfill $^*$) svn revision number refers to manual and software package.

\vfill \pagebreak

%--------------------------------------------------------------%
%            Acknowledgements                                  %
%--------------------------------------------------------------%

\addcontentsline{toc}{subsection}{Acknowledgments}

{\it Acknowledgments.} Code management for \wwt\ is provided by NCEP. Arun
Chawla provided a first filter for this report.

\vspace{\baselineskip} \noindent
This report is available as a pdf file from

\vspace{\baselineskip}
\centerline{http://polar.ncep.noaa.gov/mmab/notes.shtml}


\vfill \pagebreak
%--------------------------------------------------------------%
%            Contents                                          %
%--------------------------------------------------------------%
\addcontentsline{toc}{subsection}{Table of contents}

\tableofcontents

\pb
\pagestyle{empty}

\bpagea


\pb
\pagestyle{\pstyle}
\pagenumbering{arabic}
%--------------------------------------------------------------%
% Section : Introduction                                       %
%--------------------------------------------------------------%
\section{Introduction} \label{sec:intro}
\newsec

\noindent
This report describes a portable package of scripts and {\F fortran} programs
that has been designed to work transparently with the \wwt\ wave modeling
framework \citep[model henceforth denoted as \ws]{tol:MMAB09a} to perform the
genetic optimization \citep[e.g.,][]{bk:ES03} of the Generalized Multiple DIA
\citep[\gmd,][]{tol:MMAB10d}.  In this report, fonts as used in the \ws\
manual are used, with the {\file file, program and directory names}, {\code
code and variables in scripts and command lines}, and {\F fortran} source code
identified by the fonts used here. Familiarity with the \ws\ code and manuals
is assumed.

Section~\ref{sec:pack_desc} describes elements of the packages, and
Section~\ref{sec:pack_use} describes how to use this package.
Section~\ref{sec:pack_graph} presents several graphical tools provided with
this package.


\bpage
\pb
%--------------------------------------------------------------%
% Section : Description                                        %
%--------------------------------------------------------------%
\section{Description of package} \label{sec:pack_desc}
\newsec

The genetic optimization package for the \gmd\ assumes an implementation of
the \ws\ model in a conventional way, with access to all different wave model
executables through the default search path of the operating system. It should
be noted that the wave model needs to be set up separately for producing
either the baseline conditions to which the \gmd\ is tuned (typically a full
solution to the nonlinear interactions with the {\code !/NL2} switch), or the
actual \gmd\ model when tuning this model, or any other $S_{nl}$ approach to
be run for direct comparison.

The optimization package uses three main directories.  The first is the main
directory ({\code \$genes\_main}) that contains the scripts and files with
data necessary to run the genetic optimization of the \gmd. These are isolated
in their own directory for easy maintenance and portability of the package.
The second is the data directory ({\code \$genes\_data}) where model data are
stored. Such data include baseline data to which the \gmd\ based model is
tuned, generational information as developed inside the genetic optimization
routine, and model results for display and documentation of model behavior
through the optimization procedure.  The third is the work directory ({\code
\$genes\_work}), which provides temporary storage used during (test)
computations only. Note that most of the optimization work is performed in the
data directories in which the generational data is stored, as will be
illustrated below.

The main package as stored in the {\code \$genes\_main} directory consists of
several sets of scripts, programs and files. At the core of the package are
the actual test cases, stored in {\code \$genes\_main/tests}:

\begin{flist}
\fit{test\_01}{Deep water time-limited growth test (single point) with
               constants wind speed and direction.}
\fit{test\_0X}{like {\file test\_01} but with much larger minimum source term
               time step, used for quick and dirty assessment if GMD
               configuration is viable (not to be used in error computation).}
\fit{test\_02}{Deep water fetch-limited growth test with constant wind speed
               and direction.}
\fit{test\_03}{Deep water 'homogeneous front' case of \cite{tol:JPO92},
               Fig.~5.} 
\fit{test\_04}{Deep water homogeneous rotating wind case.}
\fit{test\_05}{Deep water slanting fetch case.}
\fit{test\_06}{Like {\file test\_01} with background swell field added.}
\fit{test\_11}{Time-limited growth is shallow water with reducing depth.}
\fit{test\_12}{Wind sea breaking on beach.}
\fit{test\_13}{Swell breaking on beach.}
\end{flist}

\noindent
For final independent testing of GMD configurations using more realistic
conditions, two ``real-world'' test have been added. These test are used only
for comparison, not in the actual optimization.

\begin{flist}
\fit{test\_hr}{A synthetic moving hurricane based on {\file
               mww3\_test\_05} as distributed with \ws.}
\fit{test\_LM}{A storm case on Lake Michigan.}
\end{flist}

\noindent
Unlike the test used for optimization, the latter two tests require ancillary
information such as two-dimensional model grids and evolving wind fields. Such
data are stored in the directories \\
\centerline{\code \$genes\_main/tests/test\_hr.data}
\noindent and \\
\centerline{\code \$genes\_main/tests/test\_LM.data}.

\vspace{\baselineskip}
\noindent
All test cases run exclusively in the work directory from which the script is
called, and all produce full spectral and source term output for a selected
set of locations and/or times (depending on the actual test). These test are
typically not run independently, but inside a scripting environment. The
corresponding scripts are found in the main directory ({\code \$genes\_main}),
which also includes setup and cleanup scripts. The following scripts are found
in this directory:

\begin{flist}
\fit{run\_setup.sh}
    {Setup basic system and system shell scripts variables such as {\code
     \$genes\_main}.} 
\fit{run\_clean.sh}
    {Clean up main and work directories. Optionally clean up programs and
     executables directories.}
\fit{run\_make.sh}
    {Compile or recompile all auxiliary programs. Note that the user needs to
     set up the compiler options inside this script.}
\fit{run\_test.sh}
    {Run a single test script and put the test results in the main directory
     for inspection.}
\fit{run\_base.sh}
    {Run a set of test as identified in {\file genes.cases.env} and store the
     test data in {\code \$genes\_data} for later use as benchmark or for
     model intercomparison.}
\fit{run\_comp.sh}
    {Process raw data files of a run to produce secondary data files for
     display and (optionally) compute errors against a baseline data set.}
\fit{run\_comp.all}
    {Run {\file run\_comp.sh} for a set of runs at once.}
\fit{control.sh}
    {General management script for genetic \gmd\ optimization.}
\fit{control.cycle}
    {Run {\file control.sh} until the optimization is finished, or until a
    problem in the optimization has stopped progress (i.e., generation of new
    generations as expected).}
\fit{control.stop}
    {Gracefully shut down {\file control.sh} and {\file control.cycle} (does
    not influence {\file control.cron} below).}
%\fit{control.cron} {Run and/or check on existence of {\file control.cycle},
%    set up to run from Unix/Linux cron environment.}
\fit{descent.sh}
    {General management script for steepest descent \gmd\ optimization,
    starting from a member of a population of the genetic optimization.}
\fit{descent\_sort.sh}
    {Convert quadruplets resulting from descent algorithm to sorted form.}
\fit{map\_it.sh}
    {Map error in parameter space by generating a regular discrete grid of
    parameters to be optimized.}
\fit{reset.sh}{Reset environment parameters and files to those selected for an
    existing experiment.}
\fit{convert\_hr.sh}{Auxiliary program to convert GrADS files at NCEP from big
    (IBM) to little endian (Linux) format, and to set up difference fields for
    hurricane test.}
\fit{convert\_LM.sh}{Idem for Lake Michigan test case.}
\end{flist}

\noindent
To assure consistency of model setup across test and run scripts, and to allow
for flexibility of operations within scripts, several setup or environment
files are maintained. These files are kept at different locations and
maintained in different ways, as indicated in Table~\ref{tab:genes_env}. 


\begin{mytab}{tbp}
\begin{center}
\mytcap{Setup and environment files used by the genetic optimization algorithm
        for the \gmd. See also Fig.~\ref{fig:genes_dirs} for storage
        locations.} \label{tab:genes_env}
\begin{tabular}{|l|l|l|l|} \hline
File   & Location  & Maintained by & Description  \\ \hline \hline
{\file .genes.env}      & {\code \$home}   & {\file run\_setup.sh}  &
Maintain basic setup \\
{\file genes.spec.env}  & {\code \$genes\_main}   & User  &
Spectral grid settings\\
{\file genes.source.env}  & {\code \$genes\_main}   & User  &
Source term settings\\
{\file genes.snl.env}  & {work dir.}   & User/scripts  &
$S_{nl}$ (\gmd) settings\\
{\file genes.cases.env}  & {\code \$genes\_main}   & User  &
Test cases to use\\
{\file genes.w\_{\it nnn}.env}  & {\code \$genes\_main}   & User  &
Weights for errors\\
  &  &  & for case {\file test\_{\it nnn}} \\
{\file genes.weights.env}  & {\code \$genes\_main}   & User  &
Default error weights \\
{\file genes.stats.env}  & {\code \$genes\_main}   & User  &
Stats for opt. \\
{\file genes.mask.env}  & {\code \$genes\_main}   & User  &
Mask for opt. pars. \\
{\file genes.expdef.env}  & {\code \$genes\_main}   & {\file control.sh}  &
Base setup of exp. \\
{\file genes.maps.env}  & {\code \$genes\_main}   & User  &
Setup for error mapping \\
{\file genes.terr.env}  & {generation dir.}   & {\file control.sh}  &
Set filter error level \\
\hline
\end{tabular}
\end{center}
\end{mytab}


The main run time scripts described above use utility scripts and {\F fortran}
codes for their operation. Utility scripts are gathered in {\code
\$genes\_main/ush}. Note that these scripts are not intended to be run
independently.

\begin{flist}
\fit{get\_cases.sh}{Evaluate the contents of file {\file genes.cases.env}.}
\fit{get\_err\_test.sh}{Get combined error for single test case.}
\fit{get\_err\_tot.sh}{Get combined error for all test cases.}
\fit{get\_err\_par.sh}{Get combined error per parameter for active test cases.}
\fit{get\_terr.sh}{Get/set filter error level.}
\fit{make\_init.sh}{Generate first population.}
\fit{make\_next.sh}{Make next generation.}
\fit{run\_thread.sh}{Master execution script for a thread in the engine.}
\fit{run\_one.sh}{Sub-script in {\file run\_thread.sh} to get the errors for a
                  single member of the population.}
\fit{thread\_start.sh}{Start the computational engine for computing errors.}
\fit{thread\_stop.sh}{Stop the computational engine for computing errors.}
\fit{thread\_wait.sh}{Wait for the computational engine to finish.}
\fit{thread\_check.sh}{Background check on health of engine.}
\fit{}{These four scripts are kept for different machine setups in the files
        {\file thread\_start.sh.}{\code \$genes\_engn} etc., and are linked to
        the above file names in the script {\file control.sh}.}
\fit{make\_maps.sh}{Creates population for error mapping..}
\fit{colorset.gs}{GrADS color table setup script.}
\fit{spec.gs}{GrADS script for plotting of spectra.}
\fit{source.gs}{GrADS script for plotting of source terms.}
\fit{1source.gs}{GrADS script for plotting of single source term or spectrum
        (including preset copies for various tests).}
\fit{map\_hr.gs}{GrADS script for map plotting for hurricane test.}
\fit{map\_LM.gs}{GrADS script for map plotting for Lake Michigan test.}
\end{flist}

\noindent
Note that for most GrADS script links are provided in the main directory for
easy interactive use.  Source codes for programs used specifically for this
package are gathered in {\code \$genes\_main/progs}, and their executables
(replacing file name extension {\file f90} with {\file x}) are stored in
{\code \$genes\_main/exe}:

\begin{flist}
\fit{constants.f90, w3timemd.f90, w3dispmd.f90, w3arrymd.f90}{}
\fit{}{Service routines from the \ws\ code used in the optimization package.}
\fit{random.f90}{Subroutines for random number generation.}
\fit{cgaussmd.f90}{Subroutines for processing normal distributions.}
\fit{qtoolsmd.f90}{Tools for quadruplet processing (subr.).}
\fit{restart\_co.f90}{Combine two restart files.}
\fit{process.f90}{Process raw data from one or two runs to get a data set that
                  is readable by, for instance Matlab (single case use), or
                  get individual error estimates (two case use).}
\fit{err\_test.f90}{Combine errors per test.}
\fit{err\_tot.f90}{Get overall error.}
\fit{err\_par.f90}{Combine errors per parameter for active tests.}
\fit{testerr.f90}{Set filter error level based on current population.}
\fit{reseed.f90}{Get a new random seed.}
\fit{initgen.f90}{Set up first generation.}
\fit{chckgen.f90}{Process generation for the purpose of computing and
                  including errors.}
\fit{sortgen.f90}{Final sorting of generation by error. Also produces the
                  clean-up population file (sorted quadruplets).}
\fit{nextgen.f90}{Make next generation.}
\fit{getmember.f90}{Extract member information from population file.}
\fit{descent{\it{N}}.f90}{Auxiliary programs for steepest descent
                          optimization.}
\fit{mapsgen.f90}{Make generation for error mapping.}
\end{flist}


\bpage
\pb
%--------------------------------------------------------------%
% Section : Description                                        %
%--------------------------------------------------------------%
\section{Using the package} \label{sec:pack_use}
\newsec

The first step of setting up the genetic optimization package is to set up the
underlying wave model, taking into account which physics options the model
\gmd\ is to be tuned to, and selecting Cartesian grid options ({\code
!/XYG})\footnote{~Except for {\file test\_LM}, which requires the {\code
!/LLG} switch.}. For this reference is made to the system manual of \ws\
\citep{tol:MMAB09a}. The second step is to install the basic package by
unpacking the tar file {\file genes.tar}. When the tar file is unpack by
executing
\begin{center}
{\file tar -xvf genes.tar}
\end{center}
\noindent
a new directory {\file ./genes} is automatically generated. This should be the
main directory {\file \$genes\_main} in the package. If another directory name
is desired, this directory name needs to be modified before the next step of
the initialization. Co-developers of \ws\ can alternatively obtain this
directory with its contents from the subversion repository at
NCEP\footnote{~The tar file is created from the ./utilities/genes\_gmd/package
directory under the trunk.}.  The basic setup of the package is finished by
executing
\begin{center}
{\file run\_setup.sh}
\end{center}
\noindent
This will take the user through an interactive process that sets several
environmental parameters for the package. This script will set the following
shell script variables:

\begin{plist}
\pit{\$genes\_main}{Main directory}
\pit{\$genes\_data}{Directory where all data are stored, including benchmark,
                    and optimization data, as well as validation data for
		    documenting model behavior.}
\pit{\$genes\_work}{Work space for scripts (scratch).}
\pit{\$genes\_expN}{Identifiers for the optimization experiment that is worked
                    on presently. These represents subdirectories under the
                    work directory (see below).}
\pit{\$genes\_base}{Base run identifier used in optimization (typically {\code
                    WRT}).}
\pit{\$genes\_engn}{Identifier for type of engine used to compute the errors
                    of a population member. The default is {\code single},
                    which uses a single threaded engine that can be used
                    anywhere. Examples of other multi-threaded engines are
                    also provided (see below).}
\end{plist}

\noindent
These variables are stored in the setup file {\file .genes.env} (see
Table~\ref{tab:genes_env}), which is used by virtually all executable scripts.
The experiment identifiers can be reset by rerunning the setup code.  Note
that the setup file can also be edited manually. The next step is to compile
all auxiliary programs used by the package by executing
\begin{center}
{\file run\_make.sh}
\end{center}
\noindent
Note that the user will need to provide the proper compiler commands and
settings in the header of this script, before executing it. Note that the
corresponding directories ({\file \$genes\_main/progs} and ({\file
\$genes\_main/exe}) can be cleaned up (listings and executables removed) by
executing
\begin{center}
{\file run\_clean.sh}
\end{center}
\noindent
and answering affirmative to the appropriate questions. The latter script
automatically cleans up the main and work directories {\file \$genes\_main}
and {\file \$genes\_work}. This completes the initial setup of the package. It
is now prudent to test some of the test cases by running the {\file
run\_test.sh} script. This script executes a single test case and puts output
including GrADS data files in the main directory {\file \$genes\_main}. For
instance, the first test case is run in this way by executing
\begin{center}
{\file run\_test.sh test\_01}
\end{center}
\noindent
Note that {\file run\_clean.sh} should be executed between test runs to assure
that the test starts with a clean environment.

After this initial testing, benchmark or baseline datasets need to be
generated. These data set can be used both in the optimization of the \gmd\
(e.g., the exact interaction), or to identify progress (e.g., the DIA). Before
the benchmark datasets can be generated, the \ws\ model has to be compiled
with the appropriate switch settings and other options as needed, and the
appropriate namelist options need to be set in the file {\file
genes.srce.env}.  Furthermore, test cases for which benchmark data needs to be
generated need to be identified in {\file genes.cases.env}. The example
versions of these files as provided with the package are internally documented
with respect to the data format in the files.  After these preparations,
benchmark data is generated by running the command
\begin{center}
{\file run\_base.sh baseID yes}
\end{center}
\noindent
For each test case, a directory {\file \$genes\_data/baseID/test\_{\it nn}} is
generated, where the files {\file log.ww3}, {\file spec.ww3}, {\file
srce.ww3}, and {\file part.ww3} are stored. These files contain spectral,
source term ($s_{nl}$) and partitioned wave data, respectively.  The first
command line argument of this script identifies the subdirectory for the
baseline datasets. The second command line parameter is optional, and
identifies if GrADS data sets are to be saved with the general baseline data.

The above commands only generate raw test data, with the exception of the
optional GrADS data that can be used to get a quick look at full
two-dimensional spectra $F(f,\theta)$ and source terms $s_{nl}(f,\theta)$.
Detailed diagnostics files ({\file all\_data.ww3} in data directories),
specifically for processing with Matlab, can be generated with the command
\begin{center}
{\file run\_comp.sh test\_01 DIA}
\end{center}
\noindent
which processes DIA results for test case {\file test\_01}. Errors for this
case against WRT results are generated by running
\begin{center}
{\file run\_comp.sh test\_01 DIA WRT}
\end{center}
\noindent
which generates an error file {\file errors.test\_01.DIA.WRT} in the work
directory. To run this script for all tests for a given model setup the
command
\begin{center}
{\file run\_comp.all none DIA WRT}
\end{center}
\noindent
can be used, generating raw data files for the DIA and WRT model runs. To
generate errors for WAM and WW3 runs against the WRT runs, this command is
executed as
\begin{center}
{\file run\_comp.all WRT WAM WW3}
\end{center}
\noindent
Note that the {\file run\_comp.sh} or {\file run\_comp.all} scripts together
with several Matlab scripts have been used to generate most of the graphics in
\cite{tol:MMAB10d}.


\begin{myfig}{tbp}
\setlength{\unitlength}{0.1mm}

\begin{center} \begin{picture}(1370,1860)(0,-1840)

\put(   0, -1840){\framebox(1370,1860)[c]{}}

\put(  20,  -60){\makebox(0,80)[l]{{\code \$genes\_data}}}
\put(   0,  -60){\line(1,0){300}}
\put(  50,  -60){\line(0,-1){300}}

\put(  50, -120){\line(1,0){50}}
\put( 120, -160){\makebox(0,80)[l]{{\file WRT}}}
%\put( 670, -160){\makebox(0,80)[l]{Base line data from {\file run\_base.sh}.}}

\put(  50, -180){\line(1,0){50}}
\put( 120, -220){\makebox(0,80)[l]{{\file WAM}}}
\put( 670, -220){\makebox(0,80)[l]{Base line data from {\file run\_base.sh}.}}

\put(  50, -240){\line(1,0){50}}
\put( 120, -280){\makebox(0,80)[l]{{\file WW3}}}
%\put( 670, -280){\makebox(0,80)[l]{Base line data from {\file run\_base.sh}.}}

\put(  50, -300){\line(1,0){50}}
\put( 120, -340){\makebox(0,80)[l]{{\file \ldots}}}
%\put( 670, -340){\makebox(0,80)[l]{Base line data from {\file run\_base.sh}.}}

\put(  50, -360){\line(1,0){50}}
\put( 120, -400){\makebox(0,80)[l]{{\code \$genes\_exp1}}}
\put( 100, -400){\line(1,0){300}}
\put( 150, -400){\line(0,-1){60}}

\put( 150, -460){\line(1,0){50}}
\put( 220, -500){\makebox(0,80)[l]{{\code \$genes\_exp2}}}
\put( 200, -500){\line(1,0){300}}
\put( 250, -500){\line(0,-1){720}}

\put( 250, -560){\line(1,0){50}}
\put( 320, -600){\makebox(0,80)[l]{{\file genes.expdef.env}}}
\put( 670, -600){\makebox(0,80)[l]{Parameters defining exp.}}

\put( 250, -620){\line(1,0){50}}
\put( 320, -660){\makebox(0,80)[l]{{\file genes.stats.env}}}
\put( 670, -660){\makebox(0,80)[l]{Statistical info.}}
\put(1300, -660){\makebox(0,80)[l]{(c)}}

\put( 250, -680){\line(1,0){50}}
\put( 320, -720){\makebox(0,80)[l]{{\file genes.mask.env}}}
\put( 670, -720){\makebox(0,80)[l]{Mask for pars. to optimize.}}
\put(1300, -720){\makebox(0,80)[l]{(c)}}

\put( 250, -740){\line(1,0){50}}
\put( 320, -780){\makebox(0,80)[l]{{\file genes.spec.env}}}
\put( 670, -780){\makebox(0,80)[l]{Spectral discretization.}}
\put(1300, -780){\makebox(0,80)[l]{(c)}}

\put( 250, -800){\line(1,0){50}}
\put( 320, -840){\makebox(0,80)[l]{{\file genes.source.env}}}
\put( 670, -840){\makebox(0,80)[l]{Source term options.}}
\put(1300, -840){\makebox(0,80)[l]{(c)}}

\put( 250, -860){\line(1,0){50}}
\put( 320, -900){\makebox(0,80)[l]{{\file genes.cases.env}}}
\put( 670, -900){\makebox(0,80)[l]{Case selection.}}
\put(1300, -900){\makebox(0,80)[l]{(c)}}

\put( 250, -920){\line(1,0){50}}
\put( 320, -960){\makebox(0,80)[l]{{\file genes.w\_{\it nn}.env}}}
\put( 670, -960){\makebox(0,80)[l]{Weight arrays.}}
\put(1300, -960){\makebox(0,80)[l]{(c)}}

\put( 250, -980){\line(1,0){50}}
\put( 320,-1020){\makebox(0,80)[l]{{\file genes.maps.env}}}
\put( 670,-1020){\makebox(0,80)[l]{Parameters for error mapping.}}
\put(1300,-1020){\makebox(0,80)[l]{(c)}}

\put( 250,-1040){\line(1,0){50}}
\put( 320,-1080){\makebox(0,80)[l]{{\file mapping}}}
\put( 670,-1080){\makebox(0,80)[l]{Population for error mapping.}}

\put( 250,-1100){\line(1,0){50}}
\put( 320,-1140){\makebox(0,80)[l]{{\file mapping.test}}}
\put( 670,-1140){\makebox(0,80)[l]{Mapped errors per test.}}

\put( 250,-1160){\line(1,0){50}}
\put( 320,-1200){\makebox(0,80)[l]{{\file mapping.pars}}}
\put( 670,-1200){\makebox(0,80)[l]{Mapper errors per par.}}

\put( 250,-1220){\line(1,0){50}}
\put( 320,-1260){\makebox(0,80)[l]{{\file gen{\it NNNN}}}}
\put( 670,-1260){\makebox(0,80)[l]{Directory for generation {\it NNNN}.}}
\put( 300,-1260){\line(1,0){300}}
\put( 350,-1260){\line(0,-1){420}}

\put( 350,-1320){\line(1,0){50}}
\put( 420,-1360){\makebox(0,80)[l]{{\file seed.env}}}
\put( 670,-1360){\makebox(0,80)[l]{Random seed for next generation.}}

\put( 350,-1380){\line(1,0){50}}
\put( 420,-1420){\makebox(0,80)[l]{{\file population}}}
\put( 670,-1420){\makebox(0,80)[l]{What it says \ldots.}}

\put( 350,-1440){\line(1,0){50}}
\put( 420,-1480){\makebox(0,80)[l]{{\file pop\_clean}}}
\put( 670,-1480){\makebox(0,80)[l]{Sorted quadruplets.}}

\put( 350,-1500){\line(1,0){50}}
\put( 420,-1540){\makebox(0,80)[l]{{\file errors.test}}}
\put( 670,-1540){\makebox(0,80)[l]{Sorted errors per test.}}

\put( 350,-1560){\line(1,0){50}}
\put( 420,-1600){\makebox(0,80)[l]{{\file errors.pars}}}
\put( 670,-1600){\makebox(0,80)[l]{Sorted error per parameter.}}

\put( 350,-1620){\line(1,0){50}}
\put( 420,-1655){\makebox(0,80)[l]{{\file descent.{\it n}}}}
\put( 670,-1660){\makebox(0,80)[l]{Steepest desc. res. for member {\it n}.}}

\put( 350,-1680){\line(1,0){50}}
\put( 420,-1720){\makebox(0,80)[l]{\ldots}}
\put( 670,-1720){\makebox(0,80)[l]{Work files and directories depending}}
\put( 670,-1780){\makebox(0,80)[l]{on progress of processing, output req.}}
\put( 670,-1840){\makebox(0,80)[l]{and hardware setup.}}

\end{picture} \end{center}

\myfcap{Layout of directories with data for the \gmd\ optimization. (c)
        identifies copy of file with file used stored in {\code
        \$genes\_main},}
        \label{fig:genes_dirs}
\end{myfig}


\vspace{\baselineskip}
\noindent
This completes the setup of the test environment. the next step is to set up
the \gmd\ model for optimization. Before this is discussed, it is important to
understand the file and directory structure used to perform the optimization
and to save the relevant intermediate and final data. The directory and file
structure is outlined in Fig.~\ref{fig:genes_dirs}. In the setup steps
performed so far, the three main directories have been defined by running
{\file run\_setup.sh}, and can be redefined by rerunning this script or by
manually editing the file {\file .genes.env} in the users home directory. The
directories with baseline information (identified here as {\file WRT}, {\file
WAM} and {\file WW3}, actual names set by used) have already been created and
filled by running {\file run\_base.sh} and either {\file run\_comp.sh} or
{\file run\_comp.all}. There are two levels of directories to hold the actual
generational, descent and error mapping data from the optimization approach
({\code \$genes\_epx1} and {\code \$genes\_exp2}). The first is intended to
hold all data for a general \gmd\ layout. For instance, all experiments with a
single component \gmd\ with a traditional quadruplet layout could be saved
under a single directory {\code \$genes\_epx1}. Several directories {\code
\$genes\_epx2} then can hold experiments for deep versus shallow optimization,
different initial optimization seeds etc. Files identified with `(c)' are kept
in the directory for documentation only, the actual files used by the package
are stored in the main directory {\code \$genes\_main}.

The main script controlling the optimization is the script {\file control.sh}.
This script is designed to incrementally execute the model optimization,
including the initial setup for the optimization process. The layout of the
script is illustrated in Fig.~\ref{fig:control}. Running the command
\begin{center}
{\file control.sh}
\end{center}
\noindent
will get the optimization procedure started, or will continue it. When
starting a new (part of an) experiment, this script should be run
interactively first to set up the necessary directories and files. If the
script is run to start a new experiment, {\file run\_setup.sh} should be run
first to set up the proper data directories.

Note that if work is to be continued on an existing experiment, the
environment of the experiment can be restored by running
\begin{center}
{\file reset.sh} {\code \$genes\_exp1} {\code \$genes\_exp2}
\end{center}
\noindent
which resets {\file .genes.env} and restores all other environment files as
used with the selected experiment.

After processing the general setup file {\file .genes.env} in the users home
directory, the first thing {\file control.sh} does is checking the existence
of the file {\file genes.expdef.env} (see Fig.~\ref{fig:genes_dirs}). If this
file does not exist, it is generated from interactive information provided to
{\file control.sh}. The setup file sets the following shell script variables:

\begin{plist}
\pit{\$genes\_nq   }{Number of representative quadruplets.}
\pit{\$genes\_npop0}{Size of initial population.}
\pit{\$genes\_npop }{Size of subsequent population.}
\pit{\$genes\_ngen }{Number of generations to be considered.}
\pit{\$genes\_seed }{Initial random seed. Note that the package contains its
                     own random number generator, making genetic optimization
		     experiments reproducible as long as the same random seed
		     is used.}
\end{plist}

\noindent
After this file is initially generated, it can only be modified by hand. The
only parameter to modify in such a way is the requested number of generations
or possibly the random seed. Modifying the other parameters will result in
failure of an ongoing optimization, and will require the removal of all
generational directories {\file gen{\it NNNN}}.

The second step is to copy in all the environment files as indicated in
Fig.~\ref{fig:genes_dirs}, or, if these files are already there, to compare
them against the corresponding files in the main directory ({\code
\$genes\_main}). This step is added to assure that the experiment is not
accidentally continued with modified settings of the \gmd\ or of the
optimization experiment. It also allows for a simple way to redo an experiment
or to expand on an experiment by copying all setup files that are expected in
the main directory (marked with '(c)' in Fig.~\ref{fig:genes_dirs}) back to
the main directory {\code \$genes\_main} (see command {\file reset.sh} as
discussed above).

The next step is to check if the experiment has been started by generating an
initial generation. If this generation is not present, the script {\file
make\_init.sh} using the program {\file initgen.x} is used to produce the
first generation. Note only the \gmd\ setup according to information in the
setup ({\file *.env} file) is produced, but that errors are not yet. To
indicate this, errors are set to 999.999.

Following this, the program {\file chckgen.x} checks the population for cases
for which the error still needs to be evaluated. For each such population
member {\file \it nnnn}, a file with the \gmd\ namelist information is created
as the file {\file snl.{\it nnnn}}. Computing the corresponding error files
{\file err.{\it nnnn}} represents the main effort of the optimization
procedure, and is performed by the computational `engine' as will be described
below. After all errors have been evaluated, the program {\file chckgen.x} is
run again to include the errors from the error files in the population data
set {\file population}. Note that in its first call in the script, {\file
chckgen.x} processes all error files {\file err.{\it nnnn}} from previously
aborted runs of {\file control.sh}, hence avoiding duplication of work already
performed.


\begin{myfig}{tbp}
\setlength{\unitlength}{0.1mm}

\begin{center} \begin{picture}(1200,935)(0,-935)

% outer box

\put(  20,  -60){\makebox(0,80)[l]{\file control.sh}}
\put(   0, -935){\framebox(1200,885)[c]{}}

% initializations

\put(  20, -130){\makebox(0,80)[l]{Process setup file {\file .genes.env}}}
\put(  20, -180){\makebox(0,80)[l]{Test or generate {\code \$genes\_exp1/\$genes\_epx2}}}
\put(  20, -230){\makebox(0,80)[l]{Process or generate {\file genes.expdef.env}}}
\put(  20, -280){\makebox(0,80)[l]{Copy or test other setup files}}
\put(   0, -275){\line(1,0){1200}}

% initial population

\put(  20, -350){\makebox(0,80)[l]{If necessary, make first generation}}
\put(1180, -350){\makebox(0,80)[r]{({\file make\_init.sh})}}
\put(   0, -345){\line(1,0){1200}}

% check population

\put(  20, -420){\makebox(0,80)[l]{Check generation, for all members do}}
\put(1180, -420){\makebox(0,80)[r]{({\file chckgen.x})}}
\put(  70, -490){\makebox(0,80)[l]{If error not processed, make file {\file
                                   snl.{\it nnnn}}}}
\put(  70, -540){\makebox(0,80)[l]{with \gmd\ setup for wave model.}}
\put(  50, -415){\line(1,0){1150}}
\put(  50, -415){\line(0,-1){120}}
\put(   0, -535){\line(1,0){1200}}

% Process errors

\put(  20, -610){\makebox(0,80)[l]{For all files {\file snl.{\it nnnn}} do}}
\put(  70, -680){\makebox(0,80)[l]{Compute error file {\file err.{\it nnnn}}}}
\put(1180, -680){\makebox(0,80)[r]{`computational engine'}}
\put(  50, -605){\line(1,0){1150}}
\put(  50, -605){\line(0,-1){70}}
\put(   0, -675){\line(1,0){1200}}

% recheck population

\put(  20, -750){\makebox(0,80)[l]{Check generation, if all errors present do:}}
\put(1180, -750){\makebox(0,80)[r]{({\file chckgen.x})}}
\put(  50, -745){\line(1,0){1150}}
\put(  50, -745){\line(0,-1){70}}

% sort population

\put(  70, -820){\makebox(0,80)[l]{Sort the population by total error and produce}}
\put(  70, -870){\makebox(0,80)[l]{other population and error files.}}
\put(1180, -870){\makebox(0,80)[r]{({\file sortgen.x})}}
\put(  50, -745){\line(1,0){1150}}
\put(  50, -865){\line(1,0){1150}}
\put(  50, -745){\line(0,-1){190}}

% start the next generation

\put(  70, -940){\makebox(0,80)[l]{Start the next generation}}
\put(1180, -940){\makebox(0,80)[r]{({\file make\_next.sh})}}
\put(   0, -935){\line(1,0){1200}}

\end{picture} \end{center}

\myfcap{Structure of the main genetic optimization routine {\file
        control.sh}. The `computational engine' is described in the
        manuscript.} \label{fig:control}
\end{myfig}

If all errors have been computed, the program {\file sortgen.x} sorts the
population by ascending error into the file {\file population}, and
furthermore generates the file {\file pop\_clean} with the same information
but with the quadruplets sorted so that $\mu \le \lambda$ and with ascending
order for $\lambda$ per quadruplet. Note that the latter sorted information is
used to eliminate duplicates in the generation of the next generation. This
program also produces the error files {\file errors.pars} and {\file
errors.test}.  Note that the latter files are saved for later analysis, but
are not used in the optimization process and/or {\file control.sh}.  Finally,
the script {\file make\_next.sh} and the program {\file nextgen.x} produce the
next population, again with dummy values for the error for new members of the
population.

This ends the description of the script {\file control.sh}. This script is not
set up to cycle through consecutive generations. The script is cycled by the
additional command
\begin{center}
{\file control.cycle}
\end{center}
\noindent
which sets up {\file genes.expdef.env} interactively as needed, and then
cycles {\file control.sh} until it works on the same generation for the third
time. The latter indicates a failure in the optimization attempt, or reaching
the required number of generations. Each call of {\file control.sh} has its
own output file {\file control.{\it nnnn}.out} for later inspection. Here
{\file {\it nnnn}} identifies the sequence number of runs of {\file
control.sh}, not the generation number.

\vspace{\baselineskip}
\noindent
This leaves the description of the computational engine for computing errors
for members of the population. This engine is designed to be able to run a set
of parallel job streams, which are dynamically fed with work to do to optimize
load balancing. At the center of the engine is the utility script {\file
run\_thread.sh}. This script manages a single thread of the computational
engine. If the hardware allows for parallel threads of computation, multiple
versions of {\file run\_thread.sh} are operating simultaneously. How this is
achieved depends on the hardware and job scheduling software used. The
starting and stopping of the threads of the engine is managed by the scripts

\begin{flist2}
\fit{thread\_start.sh}{Start a number of copies of {\file run\_thread.sh}.}
\fit{thread\_stop.sh}{Stop all copies of {\file run\_thread.sh} .}
\fit{thread\_wait.sh}{Wait for all copies of {\file run\_thread.sh} to stop.}
\fit{thread\_check.sh}{Check health of each copy of {\file run\_thread.sh}.}
\end{flist2}

\noindent
These scripts are actually links to scripts {\file thread\_start.sh,}{\code
\$genes\_engn} etc., where {\code \$genes\_engn} represents a setup of the
computational engine. Options for for the engine ({\code \$genes\_engn})
provided with the package are:

\begin{plist}
\pit{single}{A single copy of {\file run\_thread.sh} runs in the background on
             the present machine.}
\pit{snits} {Many copies of {\file run\_thread.sh} on several nodes of a Linux
             cluster (snits is Hendrik's present cluster). Note that this
             cluster approach used background runs of the script through
             {\file ssh} started from an interactive node, and does not use a
             batch scheduling system. The system is also set up to run some
             processes on the front end of the cluster if so desired.}
\pit{IBM\_ll}{Running on an IBM supercomputer with LoadLeveler as a batch
              processor. {\file control.sh} or {\file control.cycle} is run
              interactively, whereas the computational engine is submitted as
              a batch job. The package is distributed with various sizes of
              computational engines stored as {\file
              thread\_start.sh.IBM\_ll.{\it NNN}}, where {\it NNN} indicates
              the number of parallel processes in the engine. The file to be
              used is linked to {\file thread\_start.sh.IBM\_ll} to be
              activated. }
\end{plist}

\noindent
For each copy of {\file run\_thread.sh} with number {\it nn} the following
files and directories are maintained in the generation directory:

\begin{flist}
\fit{thread\_{\it nnn}}{Work directory for this copy of the script.}
\fit{thread\_{\it nnn}.out}{Output file for this copy of the script.}
\fit{tdata.{\it nnn}.out}{File used for communication between {\file
                         control.sh} and each individual copy of {\file
                         run\_thread.sh} that is running.}
\end{flist}

\noindent
All these files and directories are removed from the generation directory
after all imputations for that directory have been completed.  The file {\file
tdata.{\it nnn}} contains a single text string. Valid values of this string
are

\begin{plist}
\pit{starting}      {Initial string value set by {\file thread\_start.sh}.}
\pit{ready to go} {{\file run\_thread.sh} signaling {\file control.sh} that it
                     is ready to accept work.}
\pit{snl.{\it nnnn}}{{\file control.sh} signaling {\file run\_thread.sh} to
                      work on the case as described by the file {\file snl.{\it
                      nnnn}} in the work directory.}
\pit{done}           {{\file control.sh} signaling {\file run\_thread.sh} to
                     stop operations and end its execution.}
\end{plist}

\noindent
Finally, {\file run\_thread.sh} uses the utility script {\file run\_one.sh} to
compute the errors for the \gmd\ as described by {\file snl.{\it nnnn}}. 

\vspace{\baselineskip}
\noindent
It should be noted that a disproportionally large part of the computational
effort in the genetic optimization is lost for computing test cases with
unstable model behavior. In such cases, the dynamic time step becomes
extremely small, and hence the model run time becomes extremely long. This
computational effort is essentially waisted, because the cases will have large
errors and will hence have no impact on subsequent populations. To eliminate
such useless computations, the error from case {\file test\_01} (or {\file
test\_11} for the shallow water tests) is assessed in {\file run\_one.sh}. If
this error is too big, other test cases are skipped, and the error of {\file
test\_01} is used as a proxy for the error of running all tests.

However, even {\file test\_01} can take up much computational effort by
itself, due to the small minimum source term integration time step allowed
(1~s). To speed up initial computations for unstable cases, the test case
{\file test\_0X} has been introduced. This test case is identical to {\file
test\_01} with the exception that the minimum source term time step is set to
300~s. This test can give a `quick and dirty' assessment of the viability of
the configuration of the GMD, and will run fast, independent of the viability
of the model configuration. Note that this test is run only to check
viability. If viability is established, {\file test\_01} will be used to
compute errors for this configuration, and results of {\file test\_0X} will be
ignored. In some cases the small time step in {\file test\_01} will lead to
large model errors whereas {\file test\_0X} shows much more moderate
errors. Therefore, a second filter test is applied to the errors produced by
{\file test\_01}.

The filtering is performed in the basic computational script {\file
run\_one.sh}, which is used by all optimization methods (genetic, descent,
mapping).  The filter levels are set in the script {\file get\_terr.sh} in the
utility script directory. This script set a minimum and maximum error filter
level, and a factor used to set the error filter level based on the best
member of the previous population. If the minimum error is set to 999.999, no
filtering will be performed. The latter setting should be used if error
mapping is performed.


\vspace{\baselineskip}
\noindent 
This completes the description of the genetic optimization as performed by
{\file control.sh}. As mentioned in \cite{tol:MMAB10d}, the genetic
optimization can be augmented with a steepest descent method starting from
given members of given populations. This can be achieved by executing
\begin{center}
{\file descent.sh igen ipop}
\end{center}
\noindent
where {\file igen} represents the generation number, and {\file ipop}
represents the sequence number of the member of the sorted population from
which the steepest descent is to start. The resulting sequence of
incrementally improved configurations is stored in {\file descent.{\it igen}}
in the generation directory.  The script {\file descent.sh} uses the
computational engine designed for {\file control.sh}. Result of the steepest
descent search are saved in the generational directory {\file gen{\it nnnn}}
with {\it nnnn} corresponding to {\file igen} in the file {\file descent.{\it
nnnn}}, where {\it nnnn} corresponds to {\file ipop}. This file contains a set
of incrementally improved \gmd\ layouts, anding with the final optimal \gmd\
for the chosen initial condition.
\begin{center}
{\file descent\_sort.sh igen ipop}
\end{center}
\noindent
will sort the resulting best configuration from the file {\file descent.{\it
igen}} and store the results in the file {\file descent\_sort.{\it igen}}.
optimization,

\vspace{\baselineskip}
\noindent
Finally, the tools developed for the genetic search algorithm also make it
easy to develop a simple script to systematically map errors in parameter
space. After the optimization masks are set in {\file genes.mask.env}, and
information for discretizing the parameters spaces is set in {\file
genes.maps.env}, the mapping of errors in the selected parameter spaces can be
performed with the command
\begin{center}
{\file map\_it.sh}
\end{center}
\noindent
This command generates a population {\file mapping} representing a preset grid
in parameter space. The script then generates error for each population member
using the computational engine developed for the script {\file
control.sh}. Note that the population will not be sorted by error per
population member. Once all errors are computed, error files {\file
mapping.test} and {\file mapping.pars} are generated corresponding to the
similar two error files generated by {\file control.sh}. Note that the script
can be interrupted, and will take off where it stopped when restarted, as does
{\file control.sh}. Note, furthermore, that changing the grid of the parameter
space will require a full rerun of the script. The script could be expanded to
incrementally generate mapping in parameter space, but such an option has not
been implemented yet. Note, finally, that only a single mapping data set is
produced per directory setup.


%\bpage
\pb
%--------------------------------------------------------------%
% Section : Graphic tools                                      %
%--------------------------------------------------------------%
\section{Graphics tools} \label{sec:pack_graph}
\newsec

Two sets of graphics packages have been used for the production of graphics in
\cite{tol:MMAB10d}, and can be used with the optimization package. The first
is the GrADS package, as used as a default graphics option for the \ws\
code. Standard plotting scripts for source terms and spectra for the test
cases are described in the Section~\ref{sec:pack_desc}, and can be found in
the directory
\begin{center}
{\file \$genes\_main/ush}
\end{center}
\noindent
To use these script, links are generally made to the main directory or a work
directory, and the corresponding GrADS files are copied or linked there from
the storage area {\file \$genes\_data} or created there by the script {\file
run\_test.sh}.

\vspace{\baselineskip}
\noindent
The second package used is Matlab$^{\scriptsize \textregistered}$.  All Matlab
scripts are gathered in he directory
\begin{center}
{\file \$genes\_main/matlab}
\end{center}
\noindent 
and Matlab should be run from this directory. Under this directory is a
utilities directory
\begin{center}
{\file \$genes\_main/matlab/util}
\end{center}
\noindent 
which should be added to the default directory search path for the matlab
scripts to work properly. The utility directory contains the following utility
scripts 

\begin{flist2}
\fit{wavnu2.m}
    {Solve the dispersion relation for given frequency $\sigma$ and depth $d$.}
\fit{read\_all\_data.m}
    {Read the file {\file all\_data.ww3} for processing with Matlab.}
\fit{read\_mapping.m}
    {Read the files with error data form error mapping scripts.}
\fit{read\_pop\_clean.m}
    {Read a population file {\file pop\_clean}.}
\fit{read\_descent.m}
    {Read the files with error data form error mapping scripts.}
\end{flist2}

\noindent
The main directory for Matlab files contains the scripts

\begin{flist2}
\fit{makeplots\_base.m}
    {Make line plots for baseline cases from file {\file all\_data.ww3}.}
\fit{makeplots\_co\_5.m}
    {Make line plots for a number of baseline cases (originally 5) from file
    {\file all\_data.ww3}.}
\fit{makeplots\_maps.m}
    {Plot error maps from file {\file mapping} and {\file mapping.pars}.}
\fit{makeplots\_errors.m}{Plot error evolution as a function of generation.}


\fit{maps\_extract.sh}
    {Auxiliary script to extract mapping data from data files.}
\fit{maps\_clean.sh}
    {Auxiliary script to clean up temp files generated by {\file
    maps\_extract .sh}.}    
\fit{pop\_extract.sh}
    {Auxiliary script to extract population counts.}
\fit{pop\_clean.sh}
    {Auxiliary script to clean up temp files generated by {\file
    pop\_extract .sh}.}    
\end{flist2}

\noindent
Note that all these (Matlab) scripts require script parameters to be set
inside the scripts that point to the proper data files from the proper model
runs. The shell scripts are used to effectively extract data from data files
before reading these data into Matlab scripts.


%\bpage
\pb
%--------------------------------------------------------------%
%           References                                         %
%--------------------------------------------------------------%
\addtocontents{toc}{\vspace{\baselineskip}}
\addcontentsline{toc}{subsection}{References}
\setcounter{footnote}{0}

\bibliographystyle{jas}
%\bibliography{short,articles,books,reports,conf,mine}
\bibliography{report}

\pb
\pagestyle{empty}
\bpagea


\end{document}
